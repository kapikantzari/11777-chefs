% File project.tex
%% Style files for ACL 2021
\documentclass[11pt,a4paper]{article}
\usepackage[hyperref]{acl2021}
\usepackage{times}
\usepackage{booktabs}
\usepackage[draft]{todonotes}
\usepackage{latexsym}
% \usepackage[demo]{graphicx}
\usepackage{subcaption}
\usepackage[title]{appendix}
\usepackage{float}
\usepackage{amsmath}
\usepackage{amsfonts}
\usepackage{indentfirst}
\usepackage{xcolor}
\usepackage{natbib}


% \usepackage{subfig}
\renewcommand{\UrlFont}{\ttfamily\small}


% Packages for collaborative annotations
\usepackage{comment}
%\usepackage[draft]{todonotes}
% This is not strictly necessary, and may be commented out,
% but it will improve the layout of the manuscript,
% and will typically save some space.
\usepackage{microtype}

\aclfinalcopy 

\newcommand\BibTeX{B\textsc{ib}\TeX}

\title{11-777 Spring 2021 Class Project}

\author{
  Yun Cheng\thanks{\hspace{4pt}Everyone Contributed Equally -- Alphabetical order} \hspace{2em} Yuxuan Liu$^*$ \hspace{2em} Tiffany Ma$^*$ \hspace{2em} Erin Zhang$^*$ \\
  \texttt{\{yuncheng, yuxuanli, tma1, xiaoyuz1\}@andrew.cmu.edu}
}

\date{\today}

\begin{document}
\maketitle
\begin{abstract}
Template for 11-777 Reports using the ACL 2021 Style File 
\end{abstract}

\section{Introduction and Problem Definition}

% \begin{figure}
% \missingfigure[figwidth=\linewidth]{This is a simple example/demonstration figure that explains your task and insight}
% \end{figure}

\clearpage
\newcommand{\red}[1]{{\color{red}{#1}}}


\section{Related Work and Background}
\paragraph{Action segmentation} The goal of action segmentation is to temporarily localize action segments and classify the category of the action in each segment in an untrimmed input video. The application of action segmentation can be found in various fields such as robotics \cite{robotics_action_seg} and behavior analysis \cite{shao2012human}. Action segmentation is closely related but different from action recognition and action detection. Action recognition identifies one action in a trimmed video, and action detection usually outputs a sparse set of actions. By contrast, action segmentation is a more complex task that considers a longer range of temporal relations between sequential activities for more fine-grained action recognition in an untrimmed video.

\paragraph{Early works}
Traditional approaches generally fall into three categories: sliding window approaches, segmental models, and recurrent networks \cite{graphbased2020}. One of the earliest attempts is to detect action segments with temporal windows of different scales and non-maximum suppression \cite{6247801}. However, this method is limited by the tradeoff between larger window size and computational costs. Others use segmental models like spatiotemporal CNNs with the semi-Markov model for tracking object relationships, action transitions, and environment change \cite{lea2016segmental, 6619177}. With each action conditioned on the previous one, these methods are good at capturing local dependencies in consecutive visual patterns rather than long-range temporal relations. Other hybrid approaches include representing frames using Fisher Vectors with HMMs and GRUs for temporal modeling \cite{7477701, 8099623}, which have their main drawback of efficiency. Another line of research focuses on temporal convolutional networks (TCNs) that perform fine-grained action segmentation using temporal convolutions \cite{8099596}. The method is extended to a multi-stage architecture with a set of dilated temporal
convolutions in each stage. It is proven to be able to avoid temporal pooling and better capture long-range dependencies \cite{8953830}.  

\paragraph{Graph convolution networks} Recently, existing models are further improved by the introduction of graph convolution networks (GCNs). Built on top of action segmentation models, the graph-based module models the temporal relations between initial segmentation results with temporal proximity. It refines the pre-computed action segments by performing segment boundary regression and segment classification \cite{graphbased2020}. The latest extension to this approach constructs multi-level dilated temporal graphs for temporal reasoning at different timescales \cite{wang2020temporal}. The limitations of the graph-based module exist in its dependence on the initial backbone output. For example, it suffers from low efficiency on large graphs if the initial segmentation is heavily fragmented. However, while abundant works have been done in unimodal action segmentation, we observe that almost none of the existing work attempts at multimodal approaches. Since textual data is one of the most common and accessible annotations of video data, we are interested in incorporating texts as a complimentary domain for better segmentation results.

\paragraph{Text alignment}
\iffalse
Covered paper:
- Multimodal Machine Learning: A Survey and Taxonomy
- Learning Semantic Concepts and Temporal Alignment for Narrated Video Procedural Captioning
- Unsupervised Learning from Narrated Instruction Videos
- What’s Cookin’? Interpreting Cooking Videos using Text, Speech and Vision
\fi
Identifying the relationship between two or more modalities is one of the core challenges in Multimodal settings \cite{MultiModalSurvey}. An example of the unsupervised approaches to text alignment is to first perform temporal clustering individually on the video input and the text input, then use the two clusters to provide complementary information to one another \cite{unsupervised-align}. For instance, differences in two video segments can provide a temporal cue to a breaking point within the narrative script. Contextual information is used to assist the alignment of textual scripts and video frames \cite{temporalAlignShi}. It is built by firstly collecting a mean pooling of each modality within $K$ units. The mean representation of two modalities is then combined through a transformer model and concatenated to the embedding of the individual. Our task concerns video and scripts in the cooking domain. In one of the similar experiments, the text script is parsed into action-object (i.e. verb-noun) classes, and the video frames are aligned to the text script by matching the tokens of action-objects to those present in the frames \cite{malmaud-etal-2015-whats}.

\paragraph{Text-image matching}
To build better representation for the graph nodes, we want to use the narrations to attend to regions in the frame that are closely associated with the action that is conducted. In this way, different relevant objects in different frames can be used to distinguish the segment boundaries. 

To use attention methods, we first need to provide a set of image features. To better extract object features in the images, Faster R-CNN \cite{ren2016faster} firstly generates Region Of Interests (ROIs) with high objectness, and it then uses intermediate convolution feature maps to classify the region and regress bounding boxes. Fully Convolutional One-Stage Object Detection (FCOS) \cite{tian2019fcos} belongs to a family of anchor-less methods. Instead of regressing bounding boxes using the anchors as references, it regresses four values, $l,t,r,b$ that represent the distance from a location in the image to the four sides of the bounding boxes. Moreover, it uses CNN feature maps from different levels to perform bounding box regression at various scales to capture objects with different sizes. It has been shown that anchor-less detectors perform better than anchor-based detectors on seen and unseen test sets, and FCOS can identify objects involved in the action \cite{yoon2020semisupervised}.  

Given a set of image features, encoding regions in the image, and a set of word features extracted from the sentence, Stacked Cross Attention \cite{lee2018stacked} determines the similarity between image-sentence pair by inferring how important a region is to the sentence, and it can also reversely infer how important a sentence is to the image. An additional position feature is concatenated for the object with the visual feature extracted by ResNet \cite{wang2019position}. The image is divided into blocks, and embedding vectors representing the positions of the blocks are combined with weights determined by overlap between the block and the visual feature. The addition is motivated by the fact that the positions of objects in the image are related to the semantics of the image. This intuition aligns with our task since we expect that the relative positions of objects are associated with the action during cooking.

\clearpage

\section{Task Setup and Data (1 page)}
The main task is to segment egocentric (first-person) cooking videos from EPIC-KITCHENS dataset into action-object pairs. Given a video clip in the form of a sequence of frames, we want to identify the type of actions as well as their start and end time in the given video.


\subsection{Dataset}
% left out the word extended because we never mentioned EPIC-KITCHEN-55 %
We use the largest egocentric (first-person) dataset  EPIC-KITCHENS-100, which features 100 hours, 700 variable-length videos with 90K actions of 37 participants \cite{Damen2020RESCALING}. Compared to YouTube-based datasets such as HowTo100M \cite{miech19howto100m}, EPIC-KITCHENS contains activities that are non-scripted and thus capture more natural settings such as parallel tasking . The egocentric view provides a unique perspective on people-object interactions, attention, and intention. Meanwhile, it also imposes extra challenges compared to third-person datasets like YouCook2 \cite{ZhXuCoAAAI18}. One of the challenges is that certain actions, such as eating and drinking, cannot be directly observed due to the limited field of view. Other challenges include unseen participants, unseen cooking actions, frame noises from different sources (i.e. background and lighting), long videos with many action instances, fragmentation of segments resulted from interleaving actions in multi-tasking, and weaker temporal correlations in objects interfering the correlations in actions.

\subsection{Task formulation}
There are two input modalities: video frames of egocentric cooking scenes and narrations describing the action in the scenes. The narrations are transcribed from the audio in the form of imperative phrases: verb-noun with optional propositional phrase. The goal is to predict a verb class as well as a noun class for each frame to identify the action in the segments. Afterwards, we combine the two classes into a tuple as the final output class label. 

Formally, the visual input consists of a sequence of $M$ RGB frames in temporal order, denoted as $F=(\mathbf{f}_i)_{i=1}^M$. The RGB frames are sampled from untrimmed videos at a rate of 50 frames per second. The textual input is a sequence of $N$ audio-transcribed narrations in temporal order, denoted as $C=(\mathbf{c}_i)_{i=1}^N$. Our goal is to infer the action-object class label for each frame. The ground truth is given by $Y=(\mathbf{y}_i)_{i=1}^M$. Each $\mathbf{y}_i\in\{0,1\}^K\times\{0,1\}^L$ is a tuple of one-hot vectors encoding the true verb and noun class, where $K$ is the number of verb classes and $L$ is the number of noun classes.


\subsection{Dataset Statistics}

\subsubsection{Text Analysis}
Narrations in EPIC-KITCHENS-100 are mainly imperative phrases in the form of verb-noun with optional propositional phrase (e.g. \textit{put down plate}, \textit{put container on top of counter}). Each annotation includes start/stop frame indices. Action verbs and object nouns, which are extracted from the corresponding narration. Verbs and nouns are further classified into classes based on their semantic meaning. There are a total of 97 verb classes and 300 noun classes in the training and validation set.

We define the frequency of a verb/noun class as the number of narrations that contain a verb/noun from that class. Both verb and noun classes have a heavy tailed distribution with tail classes ($\le$ 1/15 of the maximum frequency) accounting for 13.02$\%$ and 11.67$\%$ total verbs and 5.38$\%$ and 1.85$\%$ total nouns in the training and validation set respectively (Figure \ref{fig:verb_freq}). Such a distribution indicates the intrinsic complexity and entropy of the text data. Since there were no constraints on the recording duration, we observe a great variability across videos. Average sentence length of training and validation set is 15.1 and 14.8 with standard deviation of 6.3 and 6.0 words, respectively. Average number of actions per video is 135.8 (training) and 70.1 (validation) with standard deviation of 167.7 and 93.2. More distribution statistics can be found in Appendix A Table \ref{table:train_val_stats}.

%A natural assumption of our task is that there is none or minimal overlapping between action segments, i.e. only one action in almost all time frames. We check that there are at most 4 narrations in parallel in training and 3 in validation; only 3832 (5.70$\%$) and 617 (0.92$\%$) pairs of consecutive actions overlap for more than 1 second. We also inspect the feature embeddings of the verb and noun classes. Using GloVe word vectors pre-trained on Twitter (200d vectors) \cite{pennington2014glove}, we do not notice significant interclass or intraclass clustering effect (Appendix \ref{appendix:A} Figure \ref{fig:embedding}).

% \begin{figure}[t!]
%     % \begin{minipage}[b]{1\textwidth}
%         \begin{subfigure}[b]{0.475\textwidth}
%             \centering
%             \includegraphics[scale=0.42]{figures/length_vs_actions_Training.png}
%         \end{subfigure}\\
%         \begin{subfigure}[b]{0.475\textwidth}
%             \centering
%             \includegraphics[scale=0.42]{figures/length_vs_actions_Validation.png}
%         \end{subfigure}
%         \caption{Number of action in each video against video length (in seconds)}
%         \label{fig:action-freq-video-length}
%     % \end{minipage}
% \end{figure}

\subsubsection{Video Frame Analysis}
We extract RGB frames from the videos at a sampling rate of 50 FPS. Each frame is identified by participant id, video id, and a start/end frame number. More than half of the total 700 videos in the dataset have less than 25,000 frames.  

Videos in EPIC-KITCHENS-100 have varied length with the longest video of 3708 seconds and shortest video of 10 seconds. 85.7 $\%$ of the videos are shorter than 1000 seconds and 66.0 $\%$ are less than 500 seconds (Appendix A Table \ref{table:train_val_stats}, Figure \ref{fig:duration}). 
%lacking sufficiency in support of longer videos.
% \cite{graphbased2020} mentioned that optimization tends to be slow for longer videos in action recognition tasks. 
We also see that the number of narrations grows roughly linearly with video length (Figure \ref{fig:action-freq-video-length}).  

%[Trim] If need to trim, we can trim on example words%
% The top-10 verb classes with the most number of frames include actions like \textit{grate, wait, prepare, knead, stir}, and \textit{cut}; while those with the least number of frames contain actions like \textit{bend, turn-off, turn-on, take, close}.  and the average length of the action aligns with how people would respond if asked about which action would take longer. It seems that actions involved during cooking take longer than those related to intermediate preparatory steps. 
We compile all training and validation samples of a given verb class and compute the average number of frames for this class (Appendix Figure \ref{fig:avg-frame}). For most verb classes, the average number of frames in each class are roughly the same in both training and validation set, except a few where the validation sets have more frames. We also count the total number of frames for each verb class, summed over all training and validation samples in the class. We notice that such frequency corresponds to the trend of verb-class frequency in the annotations (Appendix A Figure \ref{fig:total-frame}). This indicates that within the dataset, the frequency of the verb class correlates to the amount of visual information in the dataset.

%The dataset also provides bounding-box annotations for each frame, where it only distinguishes between two categories: hands and objects. Only active objects are annotated, so the number of object bounding-boxes in a frame approximates the number of objects that the person interacts with. We compute the average number of hand bounding-boxes appearing in a frame of each verb class. 
%Class with less than 1.5 hand bounding-boxes include actions like \textit{take, put-on, open, pull-down, walk}, and these correspond roughly to human impression on how many hands are needed for performing the action. We also compute the average number of objects bounding boxes in a frame of a given verb class. Verb classes with less than 1.8 object bounding boxes include actions like \textit{open, close, shake, check, fold}, and \textit{drink} (Appendix A Figure \ref{fig:hand-bbox}, \ref{fig:obj-bbox}). 
% \todo[inline]{Would be nice if we can generalize like more avg number of bounding boxes correlates to xx type of action?}
%The average numbers of hand and object bounding boxes for the training and validation sets are mostly equal, despite the validation set misses a few verb classes. Full details can be found in Appendix \ref{appendix:A}.




\subsection{Metrics}
%Frame-wise metrics is most commonly used in segmentation and include accuracy, precision, and recall. However, this group of metrics tends to be influenced more by actions with long duration than by those with short duration \cite{wang2020temporal}. Another problem is that it does not penalize for over-segmentation errors in the model \cite{wang2020temporal, 8099596}. Therefore, we also consider segmental edit distance, which is useful because it reflects out-of-order and over-segmentation errors \cite{ 8099596}. 

We measure our performance based on segmental F1 score, edit score, and frame wise accuracy \cite{8099596}. For each predicted action segment, we calculate its IoU with respect to the corresponding ground truth. If the score is above a threshold $\tau$, then the prediction is considered as a true positive (TP) otherwise a false positive (FP). Over-segmentation is addressed since if more than one correct segments lie within a single true action, only one is labelled as TP and all others are FP.

% We measure three metrics: frame-wise accuracy, segmental edit distance and segmental F1 score.

% Frame-wise metrics is most commonly used in segmentation and include accuracy, precision, and recall. However, this group of metrics tends to be influenced more by actions with long duration than by those with short duration \cite{wang2020temporal}. Another problem is that it does not penalize for over-segmentation errors in the model \cite{wang2020temporal, 8099596}.
% Therefore, we also consider segmental edit distance, which is useful because it reflects out-of-order and over-segmentation errors \cite{ 8099596}. 

% \cite{8099596} also introduces segmental F1 score, which not only penalizes for over-segmentation but also avoids penalizing for minor temporal shifts between the prediction and the ground truth. It also has the advantage of depending on the number of actions instead of their duration. For each predicted action segment, we calculate its IoU with respect to the corresponding ground truth. If the score is above a threshold $\tau$, then the prediction is considered as a true positive (TP) otherwise a false positive (FP). Over-segmentation is addressed since if more than one correct segments lie within a single true action, only one is labelled as TP and all others are FP. Here we consider overlapping thresholds of 10\%, 25\% and 50\%, denoted by F1$@\{10,25,50\}$. 





% We use four methods as our baseline models: both EDTCN \cite{8099596} and MSTCN++ \cite{8953830} use temporal convolution networks to capture long-range dependencies. DTGRM \cite{wang2020temporal} uses multi-level dilated temporal graphs with an auxiliary self-supervised task to help correct wrong temporal relation in videos


\clearpage
\section{Models (2 pages)}

\subsection{Baselines}

We have three baseline models: FC, MS-TCN \cite{9186840}, and DTGRM \cite{wang2020temporal}. 

%%%%%%%%%%%%% excluded opening on baseline section
% both EDTCN \cite{8099596} and MS-TCN++ \cite{9186840} use temporal convolution networks to capture long-range dependencies. \newcite{8099623} proposed a hybrid usage of GRU-based RNN and HMM to refine action alignment. DTGRM \cite{wang2020temporal} uses multi-level dilated temporal graphs with an auxiliary self-supervised task to help correct wrong temporal relation in videos.

\paragraph{FC}
We implement a vanilla 2-layer fully connected neural network that performs frame-wise classification on the input video frames. The inputs are features of dimension 1024 extracted using pretrained I3D \cite{8099985}. 

\paragraph{MS-TCN}
MS-TCN \cite{8953830} is a multi-stage architecture using TCN. The first layer of a single-stage TCN (SS-TCN) adjusts inputs dimension, followed by several dilated 1D temporal convolution layers with dilation factor doubled at each layer. All layers have ReLU activation with the residual connection. MS-TCN stacks multiple SS-TCNs so that each takes initial prediction probabilities from the previous stage and refines it. The overall architecture is trained with the cross entropy classification loss and a truncated mean squared error over the frame-wise log probabilities that penalizes over-segmentation. 

\paragraph{DTGRM}
\newcite{wang2020temporal} proposed DTGRM which refines a predicted result given by the backbone model (e.g. I3D) iteratively. The model stacks $K$ dilated graph convolution layers to perform temporal reasoning across long timescales, where each layer updates the hidden representation of every input frame. To reduce over-segmentation error, an additional self-supervised task is introduced to simulate over-segmentation error by randomly exchanging part of input frames. Both the original and exchanged frame sequences are fed into the model as input, with the output being action class likelihood for two frame sequences as well as exchange likelihood for each frame. 
% Since the model was trained on datasets with relatively shorter videos compared to EPIC-KITCHENS, we plan to trim the videos into overlapping clips of length 15 minutes with fixed fps for consistency.

%%%%%%%%%%%%%%%%%%%%%%%%%%%%%%%%% Tiff short version

% Our main baseline model is MS-TCN \cite{9186840}, which uses temporal convolution networks to capture long-range dependencies.

% \paragraph{MS-TCN}
% MS-TCN \cite{8953830} is a multi-stage architecture using TCN. The first layer of a single-stage TCN (SS-TCN) adjusts inputs dimension, followed by several dilated 1D temporal convolution layers with dilation factor doubled at each layer. All layers have ReLU activation with the residual connection. MS-TCN stacks four SS-TCNs so that each takes initial prediction probabilities from the previous stage and refines it. The overall architecture is trained with the cross entropy classification loss and a truncated mean squared error over the frame-wise log probabilities that penalizes over-segmentation. 

\subsection{Proposed Approach}
% It is relatively difficult for MSTCN to start learning the class of each input feature early: at the beginning, it naturally tries to predict the most frequent verbs in the dataset because of the use of cross-entropy loss. 
% Why Slow-Fast. 
% Following the paper [slow-fast], which proposes a two stream approach, where the fast branch tries to capture motions in the segments and getting a general sense of how objects move in the scene,  
% The intuition is that the objects in view are signature  

% Experiments on baseline models have demonstrated that one major issue with the pre-existing action segmentation methods like MSTCN on EPIC-KITCHENS dataset is to correctly classify the actions of each segment since EPIC-KITCHENS has much more diverse action classes than other datasets \cite{5995444, 10.1145/2493432.2493482, 6909500} that the baselines have evaluated on. Therefore, our proposed method utilizes MSTCN as a backbone model assisted by region of interest visual feature extraction and improves classification with an extra video-text matching component.

\subsubsection{Backbone Model}
We use the original implementation of MSTCN in \newcite{8953830} as the backbone model. The backbone takes in features extracted by I3D, same as in \newcite{8953830}. Given the feature vectors $(\mathbf{x}_1,\dots,\mathbf{x}_M)$ of a video, the model outputs an initial segmentation $(\hat{\mathbf{y}}_1,\dots,\hat{\mathbf{y}}_M)$ where $M$ is the number of frames and $\hat{\mathbf{y}}_i$ is the action class label of the predicted verb of frame $i$. From the prediction, we can generate $N'$ segments and their corresponding start-end frame number $\{(s_i,e_i)\}_{i\in [1\dots N']}$, by treating consecutive frames that are predicted with the same class as in the same segment. 

\begin{figure}[t]
    \centering
    \begin{tikzpicture}[scale=0.7]
        \filldraw[fill=black!20!white, draw=black] (-4.5,0) rectangle (-2,0.4) node[pos=0.5] {\small Narration};
        \filldraw[fill=black!20!white, draw=black] (0.5,0) rectangle (3.5,0.4) node[pos=0.5] {\small Segment};
        \draw[->,-stealth,semithick,densely dashed] (0.625,-0.6) to (1.5,0);
        \draw[->,-stealth,semithick] (2.25,-0.6) to (2,0);
        \draw[->,-stealth,semithick,densely dashed] (3.625,-0.6) to (2.5,0);
        \filldraw[fill=red!40!white, draw=black] (0,-0.6) rectangle (1.25,-1);
        \filldraw[fill=cyan!40!white, draw=black] (1.25,-0.6) rectangle (3.25,-1) node[pos=0.5] {\small Prediction};
        \filldraw[fill=blue!40!white, draw=black] (3.25,-0.6) rectangle (4,-1);
        \node [trapezium, trapezium angle=75, minimum width=2cm, draw, semithick, text width=1.2cm, align=center, fill=magenta!50!black!20!white] at (-3.25,2) {\small Text network};
        \node [trapezium, trapezium angle=75, minimum width=2cm, draw, semithick, text width=1.2cm, align=center, fill=red!40!yellow!20!white] at (2,2) {\small Video network};
        \draw[->,-stealth,semithick] (-3.25,0.4) to (-3.25,1.3);
        \draw[->,-stealth,semithick] (2,0.4) to (2,1.3);
        \draw[fill=cyan!40!black!5!white,semithick] (-0.625,4.5) ellipse (2.2cm and 1.2cm) node[text width=1.5cm,align=center] {\small Joint Embedding};
        \draw[semithick] (-0.85,3.6) circle (0.1cm);
        \draw[semithick] (-0.4,3.6) circle (0.1cm);
        \draw[->,-stealth,semithick] (-3.25,2.7) to (-0.9,3.55);
        \draw[->,-stealth,semithick] (2,2.7) to (-0.35,3.55);
    \end{tikzpicture}
    \caption{Illustration of the video-text retrieval module that computes the similarity score for a given pair of narration and video segment.}
    \label{fig:prediction-and-retrieval}
\end{figure}

\subsubsection{Video-Text Matching} \label{section:video-text-matching}
Since misclassification a more prominent issue in the baseline experiments, our proposed solution utilizes an enriched, pretrained video-text embedding to improve the labeling. We first extracted frame-level and video-level features similarly as in \newcite{miech19howto100m}. 2D features are extracted with the ImageNet pre-trained Resnet-152 \cite{7780459} at the rate of about 1 FPS, and 3D features are extracted with the Kinetics \cite{8099985} pre-trained ResNeXt-101 16-frames model \cite{8578783} to obtain about 0.78 feature per second. We freeze the ResNet and ResNeXt-101 components for feature extraction and only finetune on the final projection functions $f$ and $g$, which are composed of linear layers and gated linear units and explained in details below. 

Denote the 2D features as $(\mathbf{x}^{2D}_1,\dots,\mathbf{x}^{2D}_{M_{2D}})$ and the 3D features as $(\mathbf{x}^{3D}_1,\dots,\mathbf{x}^{3D}_{M_{3D}})$ where $\mathbf{x}^{2D}_i,\mathbf{x}^{3D}_i\in\mathbb{R}^{2048}$.
Given pairs of start-end frame number $(s_i,e_i)_{i \in [1\dots N]}$, marking the start and end of a segment, we first find the set of 2D and 3D feature indices corresponding to segment $i$ as $(s_i^{2D})_{i=1}^N$ and $(s_i^{3D})_{i=1}^N$ respectively, where $N$ is the number of segments. In other words, all 2D features with indices in $s_i^{2D}$ and 3D features with indices in $s_i^{3D}$ describe the segment from the $s_i$-th frame to the $e_i$-th frame. Then we aggregate the features of one segment using temporal maxpooling and concatenate 2D and 3D features to form a single 4096-dimensional feature vector
\begin{align*}
    \mathbf{v}^{2D}&=maxpool(\{\mathbf{x}_j^{2D}\}_{j\in s_i^{2D}})\\
    \mathbf{v}^{3D}&=maxpool(\{\mathbf{x}_j^{3D}\}_{j\in s_i^{3D}})\\
    \mathbf{v}_i&=concat(\mathbf{v}^{2D},\mathbf{v}^{3D})
\end{align*}
Similar to \newcite{miech19howto100m}, we also use the GoogleNews pre-trained word2vec embedding model to obtain a word embedding $\mathbf{c}_i$ of the text input. 
% each verb $i$ (96 $c_i$ if using actions for each action category, $977$ $c_i$ if differentiating among actions within the same action category). 
We then transform $\mathbf{v}_i,\mathbf{c}_i$ using the learned projection function finetuned on EPIC-KITCHENS $f:\mathbb{R}^{2048}\to\mathbb{R}^d,g:\mathbb{R}^{2048}\to\mathbb{R}^d$ where $d$ is the dimension of the common video-text embedding space. Finally, we perform video-text matching between a segment $\mathbf{v}_i$ and every verb $\mathbf{c}_i$ by computing the cosine similarity score as
\[s(\mathbf{v}_i,\mathbf{c}_j)=\frac{\langle f(\mathbf{v}_i),g(\mathbf{c}_j)\rangle}{\|f(\mathbf{v}_i)\|_2\|g(\mathbf{c}_j)\|_2}\]
which is high when the action $\mathbf{c}_j$ is likely to take place in the segment represented by $\mathbf{v}_i$.

In order to determine the action class of the $i$-th segment with visual feature $\mathbf{v}_i$, we calculate $s(\mathbf{v}_i,\mathbf{c}_j)$ for a set of $18003$ $\mathbf{c}_j$'s, which is the total number of possible narrations in the EPIC-KITCHENS dataset, and the word embeddings are pre-computed. We then used the action class of the $j^{*}$-th narration, where 
$j^{*} = \max_{j \in [18003]} s(\mathbf{v}_i,\mathbf{c}_j)$, to be the class prediction of the $i$-th segment. Figure~\ref{fig:prediction-and-retrieval} shows how we utilizes the joint embedding to determine the action class of each segment.

  

\subsubsection{Loss Function}
\paragraph{Backbone} We use a combination of cross-entropy classification loss
\begin{align*}
    \mathcal{L}_c&=\frac{1}{M}\sum_{t=1}^M-\log(\hat{\mathbf{y}}^*_{t,c})
\end{align*}
and truncated mean squared smoothing loss that aims to reduce over-segmentation errors as in \cite{8953830}
\begin{align*}
    \Delta_{t,c}&=|\log\hat{\mathbf{y}}^*_{t,c}-\log\hat{\mathbf{y}}^*_{t-1,c}|\\
    \Tilde{\Delta}_{t,c}&=\begin{cases}\Delta_{t,c} &\text{if $\Delta_{t,c}\le\tau$}\\\tau &\text{otherwise}\end{cases}\\
    \mathcal{L}_s&=\frac{1}{MK}\sum_{t,c}\Tilde{\Delta}^2_{t,c}
 \end{align*}
where $M$ is the number of frames, $K$ is the number of action classes, $\hat{\mathbf{y}}^*_{t,c}$ is the output probability of action class $c$ of frame $t$. We use $\tau=4,\lambda=0.15$ as in the original experiment. The final loss function is given as the sum of loss at each stage of temporal convolution
\begin{align*}
    \mathcal{L}_{stage}&=\mathcal{L}_c+\lambda\mathcal{L}_s\\
    \mathcal{L}&=\sum_{stage}\mathcal{L}_{stage}
\end{align*}

\paragraph{Video-Text Retrieval} The joint embedding is trained separately using the max-margin ranking loss as in \cite{miech19howto100m}. The loss is given by
\begin{align*}
    \sum_{i\in\mathcal{B}}\sum_{j\in N(i)}\max(0,\delta&+s_{i,j}-s_{i,i})\\
    &+\max(0,\delta+s_{j,i}-s_{i,i})
\end{align*}
where $\mathcal{B}$ is a mini-batch sample of segments-verb training pairs, $s$ is the similarity score matrix of all training pairs, $N(i)$ denotes the set of negative pairs for pair $i$ and $\delta$ is the margin. We fix $\delta=0.1$ as in the original experiment.

\subsubsection{Novelty and Challenges}
% While most existing action segmentation methods have achieved decent performance on smaller and simpler datasets, the performance decreases significantly on larger and more complex datasets like EPIC-KITCHENS. To our best knowledge, all the existing methods work purely with video input. 
Our approach is the first attempt to solve the action segmentation task of a dataset as large and complex as EPIC-KITCHENS in the multimodal setting. 
% In particular, we exploits the semantic meaning of the text annotations to improve performance.
We aim to learn a visual-textual joint embedding where the embedding of a video segment is close to the embedding of the narration describing the segment.
%Meanwhile, existing methods aim to learn better temporal relationships among frames that are close together or far away from each other. For example, MSTCN uses dilated convolution and MSTCN++ improves with two branches of dilated convolutions each with a different dilation factor; DTGRM contains layers of graphs with nodes spanning neighborhoods of different sizes. These models, by surveying frames across time and learning temporal relationships, aim to differentiate between frames from the same action versus those from different actions. 
% The intuition behind our method is that an action verb, such as \textit{take}, represents a very generic idea corresponding to a large domain of visual features with different contexts as confounding factors. For example, the visual components of a frame that is labeled as take can vary from context to context. 
% Therefore, our video-text retrieval module aims at making all video segments of \textit{take} cluster around the word \textit{take} in the joint embedding. Then it will eliminate the distractions from intraclass variation within the category \textit{take}.

Without extensive training and fine-tuning, the video-text retrieval module gives comparable result on recall metrics, R$@\{1,5,10\}$, as in the original pretrained HowTo100M model \cite{miech19howto100m}. However, 
using the joint embedding space for action segmentation is challenging because retrieving the correct text requires good initial segmentation from MS-TCN. If the output from MSTCN differs greatly from the ground truth segmentation, the result may be much less desirable. A potential solution to this issue may be to train MSTCN with better visual features to obtain a more stabilized and credible segmentation, which are discussed in Section~\ref{section:visual-feature-analysis}.
Another challenge is that the provided narrations, which we treat as captions, are not full sentences and detailed descriptions of the video, since HowTo100M \cite{miech19howto100m} worked well with several short captions like ours concatenated together, we experiment with concatenating neighboring narrations to provide more context.

% Another solution that we experiment with is to use a more meaningful annotation phrase in the form of \textit{(verb, noun)} pair in the video-text retrieval. This may impose further challenge, such as a heavier focus on clustering based on the objects rather than the actions since the objects are more easily detectable in the video. 
% For example, if we have four segments corresponding to \textit{take apple}, \textit{take banana}, \textit{put-down apple}, \textit{put-down banana}, despite ideally we want \textit{take apple} and \textit{take banana} to be closer, it is possible that \textit{take apple} and \textit{put-down apple} segments are closer since the manipulated objects \textit{apple} and \textit{banana} are more . 
% In that way, the video-image retrieval model will provide limited improvement on classification. 
% The non-descriptive nature of the annotations makes it difficult to learn a joint embedding space that separates segments of different actions.


\clearpage
\begin{table}[t]
\begin{center}
    \begin{minipage}[b]{1\textwidth}
\begin{tabular}{lrrrrrr}
\toprule
& \multicolumn{3}{c}{Train} & \multicolumn{3}{c}{Test}\\
Methods  & Acc & Edit & F1$@\{10,25,50\}$ & Acc & Edit & F1$@\{10,25,50\}$ \\
\midrule
FC  & 44.00 & 26.71 & 12.42~~22.64~~19.40 & 34.90 & 18.58 & 17.47~~13.66~~8.04\\
% EDTCN \cite{8099596} & & & & & & \\
MS-TCN \cite{8953830} & 43.52 & & & 38.65 & & \\
% RNN+HMM \cite{8099623} & & & & & & \\
DTGRM \cite{wang2020temporal} & 52.58 & & & 37.71 & & \\
\midrule
Proposed Method             & & & & \\
\bottomrule
\end{tabular}
\caption{Results of baseline models}
\label{table:results}
\end{minipage}
\end{center}
\end{table}
\section{Results (1 page)}
The columns above are just examples that should be expanded to include all metrics and baselines.

\clearpage
\section{Analysis (2 pages)}
In this section, we will analyze the baseline models. In particular, since the baselines have not been evaluated on EPIC-KITCHENS previously, we conduct several ablation experiments to access the complexity of our dataset compared to other benchmark datasets used in the original papers of these models.
\subsection{Analysis of Baselines}

\paragraph{Datasets} 
EPIC-KITCHENS is the largest egocentric dataset. Compared to other egocentric datasets such as GTEA \cite{5995444} and cooking datasets such as 50Salads \cite{10.1145/2493432.2493482} and Breakfast \cite{6909500}, EPIC-KITCHENS contains much longer videos and richer verb classes. In addition, while 50Salads  mostly contains consecutive actions, actions in EPIC-KITCHENS are usually separated by background frames where no action is being performed. These background frames comprise a significant portion of all frames (about 32\%) and affect the performace of baseline models, especially the simpler ones. 

\paragraph{FC}
A 2-layer fully connected network is trained with batch size 16 for 50 epochs. The input frames are down-sampled to 1.25FPS. Since the classification is performed frame-wise and considers no temporal relations, the result is highly fragmental. We further note that the model tends to overfit at an early stage. The poor generalizability is indicated by the relatively low Edit score and Figure \ref{fig:baseline_qualitative}. %TODO: include graph. 

% $26.5\%$ of them were background and $13.5\%$ of them were the wash verb class. %
\paragraph{MSTCN} MSTCN demonstrates its effectiveness in segmenting out the most frequent label classes. The top 5 most frequent labels in the training set in EPIC-KITCHENS are the background, \emph{wash}, \emph{take}, \emph{put}, and \emph{cut} class. We observe that the model assigns one of the most frequent verb classes when it struggles to label the action classes. The result implies that the model is able to memorize the label frequency. One potential way to alleviate this situation is to normalize the frequency through the positional weights supplied to each class label.

EPIC-KITCHENS is more complex than the benchmark datasets in many aspects, such as longer video durations and thus more actions involved. We accounted for this increase in complexity by using 15-layer single-stage TCNs rather than the 10-layer ones which are claimed to achieve optimal performance in the original experiment \cite{8953830}. Our experiment shows that there is an improvement in when using a more complex model.

\paragraph{DTGRM} The DTGRM model builds off MSTCN by adding an additional fine-tuning component that refines segmentation around the boundaries. Similar to MSTCN, DTGRM is able to output reasonable segmentation results. \ref{fig:baseline_qualitative} shows that one of its improvements from MSTCN is its capability in clearly segmenting out smaller segments, which proves the effectiveness of the additional fine-tuning component even on a more complex dataset like EPIC-KITCHENS. However, we also notice that DTGRM tend to over-segment on videos with fewer segments.

DTGRM also suffers from the label class imbalance problem. Similar to MSTCN, it labels majority of the segments as the most frequent vocabulary classes in the training set. With the weighted loss, DTGRM is able to predict a wider variety of labels. However, the classification accuracy is not as high as in the original experiment \cite{wang2020temporal}. This is reasonable given the richer vocabulary classes available in EPIC-KITCHENS.

Due to limitations on hardware, we are not able to expand the number of layers in the DTGRM model. To accommodate this, we sampled the feature inputs for every 10 frames of input to decrease input size. The more complex, sub-sampled DTGRM improved the over-segmentation issue in the original DTGRM by avoiding overly short segmentation under a sub-sampled setting.

\paragraph{Comparison Between Models}
The FC model is able to quickly gain performance at the start of training, but the performance also saturates early on. This indicates that the dataset and task requires a more complex model to learn. We observed that both the performance of MSTCN and DTGRM suffered from unbalanced verb class labels. This is a problem introduced by the EPIC-KITCHEN dataset, increasing the difficulty of the action segmentation task. 

DTGRM has shown better performance in segmenting short action instance. However, on long action instances, DTGRM does not perform as well as MSTCN as the predictions of DTGRM tends to be heavily over-segmented.
%Dataset having unequal distribution of labels
%Learning the labels instead of learning the actual task
%Interesting point is that it learns backgrounds well, which could imply it knows "when" an action is taking place, just don't know "what" it is, so it guesses the most frequent action?%

\begin{figure*}[ht!]
\begin{center}
    \begin{minipage}[b]{1\textwidth}
        \begin{subfigure}[b]{0.475\textwidth}
            \centering
            \includegraphics[scale=0.12]{figures/P26_39-comparison.png}
            \caption{P26\_39}
            \label{fig:2639}
        \end{subfigure}\quad
        \begin{subfigure}[b]{0.475\textwidth}
            \centering
            \includegraphics[scale=0.12]{figures/P16_04-comparison.png}
            \caption{P16\_04}
            \label{fig:1604}
        \end{subfigure}
        \caption{Qualitative results of two videos across all models.}
        \label{fig:baseline_qualitative}
    \end{minipage}
\end{center}
\end{figure*}

\paragraph{Metrics} 
We experimented with two variants of loss function: cross-entropy loss with and without weighting. Since background frames comprise a sizable portion, one phenomenon we observe during training is that models tend to classify most of the frames as background, at least in the first few iterations. Down-weighting background classes can mitigate this issue, but inconsistency still exists between the loss function and metrics we used, since classifying frames as the most common verb class (e.g. background, \emph{wash}) can quickly increase frame-wise accuracy until some threshold (usually the percentage of these common verb class). Edit score is a better reflection of the fragmentation issue. Models like MSTCN that incorporate temporal information tend to have higher Edit score than simpler models such as FC.  


\subsection{Ablations and Their Implications}

From Tabel \ref{table:howto100m_label}, we see that using video-text matching to predict the action class of a given segment outputted by MS-TCN gives worse performance. By comparing the results in Table~\ref{table:howto100m_seg_threshold} and Table~\ref{table:howto100m_visual_seg_threshold}, we see that although enhancing the text input with context (\textit{Narration+Context}) boosts the matching performance considerably, maxpooling across frames of different actions ($l_v >1$) worsen the performance, comparing to \textit{Narration+Context}. 
Moreover, we have observed earlier matching on longer segments, $l_s \geq 3$, performs better, but this behavior is not consistent when $l_v > 1$, since from Table ~\ref{table:howto100m_visual_seg_threshold}, when $l_v=3,l_s=3$, the median rank is much higher than $l_v=3,l_s=1$. The learnt joint-embedding is very brittle when it comes to long video segments containing multiple actions. The high performance of \textit{Narration+Context} suggests that although text input assists in identifying the action, the quality of the visual feature is equally important in ensuring the correct retrieval. 

Another issue comes from inferring the action class after video-text matching is done. We only know that the ground-truth paired narration is among the top-matching narrations, and majority-voting among top-matching narrations does not produce consistently the correct narration, which hurts performance.

% \subsubsection{Video length}
% Variation of video length for EPIC-KITCHENS dataset is much higher than that of other datasets. The maximum video length is more than 1 hour while average is only 9 minute. To determine how video length affects performance of models, we divided the whole dataset into long videos that are longer than 10 minutes and short videos that are shorter than 10 minutes.  

% \subsubsection{Vocabulary size}
% EPIC-KITCHENS contains a total of 97 verb classes, which is much more than that of 50Salads and Breakfast, which contain 17 and 48 verb classes respectively. To see if richer verb class will affect the performance of the baseline models that are originally proposed for 50Salads and Breakfast, we select a subset from 97 verb classes containing 54 verb classes with higher correlation to cooking. The remaining verb classes are labeled as background. 
% % The training and testing result on the subset is shown in Table \ref{}.

% % TODO: include table/graph

\subsection{Qualitative Analysis and Examples}
% This section should likely contain a table of examples demonstrating how the current approach succeeds/fails.

From \ref{fig:var_baseline} we can see that it is very challenging for FC to get the class label correctly, as it predicts the most common verb ``wash" (olive-green) instead of the ground truth ``mix" (hot pink). For videos with a few number of long segments, DGTRM tends to over-segment. In \ref{fig:mstcn_joint} we see that the joint embedding trained with \textit{Narration+Context} gives a slightly better classification result than one trained only with \textit{Narration}, suggesting richness in textual modality is key.





% Please use 
\bibliographystyle{acl_natbib}
\bibliography{references}

\clearpage

\begin{appendices}
\section{Data Analysis}
\label{appendix:A}
\begin{minipage}{1\textwidth}
    In this section, we present the full details of our data analysis.
\end{minipage}
\begin{table}[ht!]
\begin{minipage}{1\textwidth}
\begin{center}
{\small
\begin{tabular}{lrrrrrrrr}
\toprule
& \multicolumn{4}{c}{Training} & \multicolumn{4}{c}{Validation}\\
~ & Max. & Min. & Avg. & Std. & Max. & Min. & Avg. & Std. \\
\midrule
Verb class frequency & 14848 & 73 & 1314 & 2829 & 1937 & 71 & 191 & 398\\
Noun class frequency & 3617 & 178 & 724 & 655 & 430 & 25 & 108 & 92\\
Sentence length & 77 & 3 & 15.1 & 6.3 & 71 & 3 & 14.8 & 6.0\\
Actions per video & 940 & 1 & 136 & 168 & 564 & 3 & 70 & 93\\
Frames per verb class & 2129212 & 20165 & 225170 & 408408 & 407425 & 2702 & 42016 & 76950 \\
Video length & 3708 & 10 & 543 & 645 & 1969 & 11 & 344 & 377 \\
\bottomrule
\end{tabular}}
\caption{Statistics of EPIC-KITCHENS-100 training and validation set}
\label{table:train_val_stats}
\end{center}
\end{minipage}
\end{table}

\begin{figure}[hp]
    \begin{minipage}{1\textwidth}
    \centering
        \includegraphics[scale=0.3]{figures/noun_count.png}
    \caption{Frequency distribution of 50 most frequent noun classes in training and validation set}
    \label{fig:noun-freq}
    \end{minipage}
\end{figure}
\begin{figure}[htp!]
    \begin{minipage}{1\textwidth}
        \begin{center}
            \includegraphics[scale=0.36]{figures/Actions_embeddings_Training.png}
            \\
            \vspace{5mm}
            \includegraphics[scale=0.36]{figures/Objects_embeddings_Validation.png}
            \caption{Example of visualizing feature embeddings of verb and noun classes in 2D and 3D space}
            \label{fig:embedding}
        \end{center}
    \end{minipage}
\end{figure}
\clearpage

\begin{figure}[ht!]
    \begin{minipage}[b]{1\textwidth}
        \begin{subfigure}[b]{0.475\textwidth}
            \centering
            \includegraphics[scale=0.4]{figures/video_length_Training.png}
        \end{subfigure}
        \begin{subfigure}[b]{0.475\textwidth}
            \centering
            \includegraphics[scale=0.4]{figures/video_length_Validation.png}
        \end{subfigure}
        \caption{Distribution of video length (in seconds)}
        \label{fig:duration}
    \end{minipage}
\end{figure}

\begin{figure}[htp!]
    \begin{minipage}[b]{1\textwidth}
        \centering
        \includegraphics[scale=0.38]{figures/avg_number_Hand_bbox_verb_class.png}
        \caption{Average number of hand bounding-boxes in each frame of given verb class}
        \label{fig:hand-bbox}
    \end{minipage}
\end{figure}

\begin{figure}[htp!]
    \begin{minipage}[b]{1\textwidth}
        \centering
        \includegraphics[scale=0.38]{figures/avg_number_Object_bbox_verb_class.png}
        \caption{Average number of object bounding-boxes in each frame of given verb class}
        \label{fig:obj-bbox}
    \end{minipage}
\end{figure}

\begin{figure}[htp!]
    \begin{minipage}[b]{1\textwidth}
        \centering
        \includegraphics[scale=0.38]{figures/avg_number_frames_verb_class.png}
        \caption{Average number of frames in a narration of a given verb class in training and validation set}
        \label{fig:avg-frame}
    \end{minipage}
\end{figure}

% \begin{figure}[htp!]
% \begin{minipage}[b]{1\textwidth}
%     \begin{minipage}[b]{0.475\textwidth}
%         \begin{subfigure}[b]{0.475\textwidth}
%             \centering
%             \includegraphics[scale=0.28]{figures/histogram_num_frame_videos.png}
        % \end{subfigure}
        % \begin{subfigure}[b]{0.475\textwidth}
        %     \centering
        %     \includegraphics[scale=0.3]{figures/avg_number_frames_verb_class_4.png}
        % \end{subfigure}
    %     \caption{Distribution of number of frames in each video}
    %     \label{fig:frame-video}
    % \end{minipage}
    % \hfill
    % \begin{minipage}[b]{0.475\textwidth}
    %     \begin{subfigure}[b]{0.475\textwidth}
    %         \centering
    %         \includegraphics[scale=0.15]{figures/total_frame_word.png}
        % \end{subfigure}
        % \begin{subfigure}[b]{0.475\textwidth}
        %     \centering
        %     \includegraphics[scale=0.3]{figures/avg_number_frames_verb_class_4.png}
        % \end{subfigure}
%         \caption{Distribution of number of frames in each video}
%         \label{fig:total-frame}
%     \end{minipage}
% \end{minipage}
% \end{figure}

% \begin{figure}[htp!]
%     \begin{minipage}[b]{1\textwidth}
%         \begin{subfigure}[b]{0.475\textwidth}
%             \centering
%             \includegraphics[scale=0.1]{figures/total_frame_word.png}
        % \end{subfigure}
        % \begin{subfigure}[b]{0.475\textwidth}
        %     \centering
        %     \includegraphics[scale=0.3]{figures/avg_number_frames_verb_class_4.png}
        % \end{subfigure}
%         \caption{Histogram of the number of frames in a video}
%         \label{fig:total-frame}
%     \end{minipage}
% \end{figure}

\begin{figure}[htp!]
\begin{minipage}[b]{1\textwidth}
    \centering
    \includegraphics[scale=0.31]{figures/total_frame_word.png}
    \caption{Distribution of number of frames in each video}
    \label{fig:total-frame}
\end{minipage}
\end{figure}

\clearpage

\begin{figure}[t]
    \centering
    \includegraphics[scale=0.25]{figures/verb_count.png}
    \caption{Frequency distribution of 50 most frequent verb class in training and validation set}
    \label{fig:verb_freq}
\end{figure}

% \subsubsection{Running MSTCN with SlowFast Features} \label{appendix:slowfast}
% We slide 

\begin{figure}[t!]
    % \begin{minipage}[b]{1\textwidth}
        \begin{subfigure}[b]{0.475\textwidth}
            \centering
            \includegraphics[scale=0.42]{figures/length_vs_actions_Training.png}
        \end{subfigure}\\
        \begin{subfigure}[b]{0.475\textwidth}
            \centering
            \includegraphics[scale=0.42]{figures/length_vs_actions_Validation.png}
        \end{subfigure}
        \caption{Number of action in each video against video length (in seconds)}
        \label{fig:action-freq-video-length}
    % \end{minipage}
\end{figure}


\subsubsection{Improve Video-Text Matching with Cross-Modal Attention}
The above describes a dual encoder model that independently maps text and video to a joint embedding. It has the advantage in scalability as it can results in efficient evaluation during test time. However, as \newcite{miech2021thinking} points out, it has limited accuracy since the simple dot product is unlikely to capture the complex vision-text interactions. Analogous to how human perform video-text retrieval, one solution is to roughly select a few promising candidates then do fine-grained search for the best candidate by paying more \emph{attention} to visual details. Therefore, we adapt the \emph{Fast} and \emph{Slow} models of \newcite{miech2021thinking} in which the \emph{fast} dual encoder quickly eliminates candidates with low relevance while the \emph{slow} cross-attention model improves retrieval performance with grounding. Given an input segment $\mathbf{v}_i$, we perform retrieval by searching for an action class $\mathbf{c}_j$ such that segment $\mathbf{v}_i$ is most likely to match action class based on the joitnn embedding $\mathbf{c}_j$. Specifically, given segment and action class pair $(\mathbf{v}_i, \mathbf{c}_j)$, we compute their similarity by \[
    h(\mathbf{v}_i, \mathbf{c}_j) = \log (p(\mathbf{c}_j|\phi(\mathbf{v}_i);\theta))
\]
where $\phi(\mathbf{v}_i)$ is extracted feature of segment $\mathbf{v}_i$ and $\theta$ is the parameters of the transformer model. To combine results from dual encoder model and cross-attention model, given input segment $\mathbf{v}_i$ and action class set $\mathcal{C}$ containing $K$ action classes. we first obtain a subset of $m$ action classes $\mathcal{C}_m$ (where $m \ll K$) that have the highest score according to the fast dual encoder model. We then retrieve the final top ranked action class by re-ranking the candidates using the cross attention model:
\[
    \mathbf{y}^*_i=\text{argmax}_{\mathbf{c}_j\in \mathcal{C}_m} h(\mathbf{v}_i, \mathbf{c}_j) + \beta s(\mathbf{v}_i,\mathbf{c}_j)
\]
where $\beta$ is a positive hyper-parameter that weights the output scores of the two models. We output $(\hat{\mathbf{y}}^*_{i,c})$ as the classification probability of frame $i$ as action $c$ based on the similarity score and $(\mathbf{y}^*_i)_{i\in s_i^{3D}}$ as new labels for segment $i,i\in[t]$.


\end{appendices}



\end{document}
